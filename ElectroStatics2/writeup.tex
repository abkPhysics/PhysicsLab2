\documentclass[12pt]{article}
\usepackage{graphicx}
\title{ElectroStatics 2}
\author{Britteny,Kishaun,Abhi}

\begin{document}
\begin{titlepage}
\begin{center}
	\textsc{\LARGE Rutgers University}\\[1.5 cm]
    \textsc{\Large Physics 2 lab}\\[0.5cm]
    \rule{\linewidth}{0.5mm} \\ [.4 cm]
    {\huge \bfseries ElectroStatics 2}\\[.4 cm]
    \rule{\linewidth}{0.5mm} \\ [1.5 cm]
    \begin{minipage}{0.4\textwidth}
	\begin{flushleft} \large
	\emph{Authors:}\\
	Brittney,Kishan,Abhi
	\end{flushleft}
	\end{minipage}
	\begin{minipage}{0.4\textwidth}
	\begin{flushright} \large
	\emph{Teacher:} \\
	James
	\end{flushright}
	\end{minipage}\\[2 cm]
	\textsc{ \Large Signatures} \\[1.7 cm] 
	\rule{10 cm}{0.5mm} \\ [2.0 cm]
	\rule{10 cm}{0.5mm} \\ [2.0 cm]
	\rule{10 cm}{0.5mm}
	\vfill
	{\large {July 19, 2013}}
\end{center}
\end{titlepage}

\section*{Part 1 - Amount of Charge}

\subsection*{procedure and goal}
	
	 We applied varying voltage sources to a spherical capacitor. We then measured the charge on the spherical capacitor.Our goal was to find the relationship between the voltage applied and charge measured. 
	 \begin{figure}[h]
	 \centering
	 \includegraphics[scale = .4]{diagram1}
	 \caption{the setup to get current to the sphere}
	 \end{figure}


	 The harry potter wand is just there to transfer current to the sphere. It is not necessary but it is safer for students ;We could have connected the voltage source straight to the sphere. 

\subsection*{Prediction}
	From our knowledge of physics we know that charge travels from high potential to low potential. The \emph{voltage source} always has a fixed potential, while the Sphere has zero potential. But after the connection from the voltage source to the sphere is made, charge travels from the voltage source to the sphere until the sphere's potential equals the voltage source's. Because we know the capacitance and final potential of the sphere, we can calculate the amount of charge on the sphere Using the the formula $C = \frac{Q}{V}$ 
	\begin{eqnarray}
	C & = & 4\pi\epsilon_0R = 4 \pi *8.85*10^{-12} *.045 = 5*10^-12\\
	Q &= & V * C = \mathbf{750} * 5 * 10^{-12} = 3.75 \ nC \\
	Q &= & V * C = \mathbf{1500} * 5 * 10^{-12} = 7.5 \ nC \\
	Q &= & V * C = \mathbf{3000} * 5 * 10^{-12} = 15 \ nC \\
	Q &= & V * C = \mathbf{6000} * 5 * 10^{-12} = 30 \ nC 
	\end{eqnarray}
	making a table we get
	\begin{quote}
	\begin{tabular}{|r|r|}
	\hline 
	voltage(v) & charge(nC) \\
	\hline 
	750 & 3.75 \\
	1500 & 7.5 \\
	3000 & 15 \\
	6000 & 30 \\
	\hline
	\end{tabular}
	\end{quote}

\subsection*{Results}
	Here are our results. The charge values here are the peak charged observed:
	\begin{quote}
	\begin{tabular}{|r|r|r|}
	\hline 
	voltage(v) & charge-predicted(nC) & charge-observed(nC) \\
	\hline 
	750 & 3.75 & 1.9 \\ 
	1500 & 7.5 & 2.9\\
	3000 & 15 & 4.1 \\
	6000 & 30 & 5.8 \\
	\hline
	\end{tabular}
	\end{quote}
\subsection*{Analysis}
The measured charge was a lot lower than the predicted. Here are possible explanations
\begin{enumerate}
	\item 
	We did not hold the connection form the battery to the sphere long enough. Capacitors take time to charge up. We held it for about 2 seconds. Not sure if that is long enough?
	\item Charge was lost to the Air. Electrostatic studies are difficult in humid environments because objects discharge quickly through multiple paths 
	\item We did not calculate the capacitance correctly. We calculated the capacitance to be $5 * 10^{-12}$ When we measured the radius the error was probably $\pm .02 \ meters$, which is quite a lot in our experiment.
\end{enumerate}
The predicted results are a perfect linear relationship, but the observed results seem to form a logarithmic relationship between voltage and charge.The best fit curve was $Q=1.86*log(V)-10.58$. We tried linear and polynomial models as well but the logarithmic had the lowest error That is very interesting.Perhaps the air finds easier to absorbs high levels of charge? The only sure conclusion we can make is more voltage $\rightarrow$ more charge.

\section*{Part 2 - Distribution of charge}
\subsection*{procedure and goal}
After we we "charged" the conducting sphere, we collected charge from different parts of the sphere. Our goal was to see if charge in one part of the sphere was different than other parts of the sphere.
\subsection*{prediction}
We predicted that the charge would be the same on every part of the sphere.At first the only place with charge would be the area where the harry potter wand touched the sphere, but since the other parts of the sphere have lower potential, the charge from the touched part would travel to the other parts of the sphere. Eventually the charge would balance itself out. 
\subsection*{Results} 
Unfortunately we have no data, it was lost in transaction.   

\section*{Part 4 - }

\end{document}